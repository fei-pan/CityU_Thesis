% !TeX program = xelatex
\documentclass[a4paper,12pt,twoside,PageStyleIII,times,custommargin,custombib]{PhDThesisPSnPDF}

% ******************************************************************************
% ******************************* Class Options ********************************
% ******************************************************************************

% `a4paper' (default): International A4 size paper, default option.
%
% `11pt' or `12pt'(default): Font Size.
%
% `oneside' or `twoside'(default): Printing double side (twoside) or single
% side. Printed copy of thesis should be typed on both sides of the paper.
%
% `print': Use `print' for print version with appropriate margins and page
% layout. Leaving the options field blank will activate Online version.
%
% ************************* Custom Page Margins ********************************
%
% `custommargin`: Use `custommargin' in options to activate custom page margins,
% which can be defined in the preamble.tex. Custom margin will override the pri-
% nt/online margin setup.
%
% Margins at the binding edge should not be less than 40mm and other margins sh-
% ould not be less than 20mm.
%
% *********************** Choosing the Fonts in Class Options ******************
%
% `times' : Times font with math support.
%
% `fourier': Utopia Font with Fourier Math font (Font has to be installed)
%            It's a free font.
%
% `customfont': Use `customfont' option in the document class and load the
% package in the preamble.tex
%
% default or leave empty: `Latin Modern' font will be loaded.
%
% ********************** Choosing the Bibliography style ***********************
%
% `authoryear': For author-year citation eg., Krishna (2013)
%
% `numbered': (Default Option) For numbered and sorted citation e.g., [1,5,2]
%
% `custombib': Define your own bibliography style in the `preamble.tex' file.
%              `\RequirePackage[square, sort, numbers, authoryear]{natbib}'.
%              This can be also used to load biblatex instead of natbib
%              (See Preamble)
%
% **************************** Choosing the Page Style *************************
%
% `default (leave empty)': For Page Numbers in Header (Left Even, Right Odd) and
% Chapter Name in Header (Right Even) and Section Name (Left Odd). Blank Footer.
%
% `PageStyleI': Chapter Name next & Page Number on Even Side (Left Even).
% Section Name & Page Number in Header on Odd Side (Right Odd). Footer is empty.
%
% `PageStyleII': Chapter Name on Even Side (Left Even) in Header. Section Number
% and Section Name in Header on Odd Side (Right Odd). Page numbering in footer.
%
% `PageStyleIII': Chapter Name on Even Side (Center) in Header. Section Number
% and Section Name in Header on Odd Side (Center). Page numbering in footer.

% ******************************************************************************
% ********************************** Preamble **********************************
% ******************************************************************************

% Contains packages and user-defined commands and settings

% ****************************** Custom Margin *********************************

% Add `custommargin' in the document class options to use this section
% Set {innerside margin / outerside margin / topmargin / bottom margin}  and
% other page dimensions
\ifsetCustomMargin
  \RequirePackage[left=37mm,right=30mm,top=35mm,bottom=30mm]{geometry}
  \setFancyHdr % To apply fancy header after geometry package is loaded
\fi

% Add spaces between paragraphs
%\setlength{\parskip}{0.5em}
% Ragged bottom avoids extra whitespace between paragraphs
\raggedbottom
% To remove the excess top spacing for enumeration, list and description
%\usepackage{enumitem}
%\setlist[enumerate,itemize,description]{topsep=0em}

% ******************* Fonts (like different typewriter fonts etc.)*************

% Add `customfont' in the document class option to use this section
\ifsetCustomFont
  % Set your custom font here and use `customfont' in options. Leave empty to
  % load computer modern font (default LaTeX font).
  
%   \setmainfont{Times New Roman}
%   \setsansfont{Arial}
%   \setmonofont{Courier New}

	\setmainfont{FreeSerif} % Contains Cyrillic (Russian letters)
	\setsansfont{FreeSans}
	\setmonofont{FreeMono}
	
%   \usepackage{newtxtext}
%   \usepackage{newpxtext}
    \usepackage{textcomp} % Contains a few common symbols
\fi

% ***************************** Better enumeration ****************************
\usepackage{enumitem}

% **************************** Custom Packages ********************************
\usepackage{pdfpages}       % Include PDF documents in LATEX
\usepackage{multicol}       % Intermix single and multiple columns
\usepackage{hologo}         % Insert LaTeX-related logos
\usepackage{fontawesome5}   % Insert some interesting vector icons 
\usepackage{academicons}    % Insert some vector icons applicable to academia 
    \definecolor{orcidlogocol}{HTML}{A6CE39}

% ************************* Microtype and SI Units ****************************
\usepackage[super]{nth}     % Generate English ordinal numbers
\usepackage{microtype}      % Subliminal refinements towards typographical perfection
\usepackage{siunitx}
    \sisetup{detect-all, math-micro = \text{µ}, text-micro = µ, per-mode = symbol, range-phrase = --}

% ********************Captions and Hyperreferencing / URL **********************
\usepackage[labelsep=space,tableposition=top,font=small]{caption}
\usepackage[caption=false]{subfig} % currently provides better vertical spacing than subcaption, dated on 3 February 2021
\usepackage[capitalise,nameinlink]{cleveref} % Referencing without need to explicitly state fig /table

% *************************** Graphics and figures *****************************
% Uncomment the following two lines to force Latex to place the figure.
% Use [H] when including graphics. Note 'H' instead of 'h'
\usepackage{float}
\usepackage[framemethod=tikz]{mdframed} % insert text with a framed box

% ********************************** Tables ************************************
\usepackage{booktabs} % For professional looking tables
\usepackage{multirow}
\usepackage{longtable}
\usepackage{tabularx}
\usepackage{diagbox}    % 表格单元格里画对角线
\usepackage{xltabular}
\usepackage{threeparttable}
\usepackage[referable]{threeparttablex} % 跨页表格longtable的threeparttable
\usepackage[twoside]{rotating}

% ************************ Formatting / Footnote *******************************
% Don't break enumeration (etc.) across pages in an ugly manner (default 10000)
\clubpenalty=500
\widowpenalty=500

\usepackage[perpage]{footmisc} % Range of footnote options
\usepackage{fancyvrb}

% ************************* Algorithms and Pseudocode *************************
\usepackage{algorithm}\floatstyle{plaintop}
\usepackage{algpseudocode}
\makeatletter
\@addtoreset{algorithm}{chapter}% algorithm counter resets every chapter
\makeatother
\renewcommand{\thealgorithm}{\thechapter.\arabic{algorithm}}% Algorithm # is <chapter>.<algorithm>

\usepackage[chapter,newfloat]{minted}
\newenvironment{code}{\captionsetup{type=listing}}{}
\SetupFloatingEnvironment{listing}{name=Listing} % or name=Source Code
\usepackage{csquotes}

% *************************** Bibliography and References ********************
% Add `custombib' in the document class option to use this section
\ifuseCustomBib
% If you would like to use biblatex for your reference management, as opposed to the default `natbibpackage` pass the option `custombib` in the document class. Comment out the previous line to make sure you don't load the natbib package. Uncomment the following lines and specify the location of references.bib file

\RequirePackage[backend=biber, date=year, style=numeric, citestyle=numeric-comp, sorting=none, natbib=true, backref=true, maxbibnames=99, autolang=hyphen]{biblatex}
\addbibresource{References/ref_1.bib} %Location of references.bib only for biblatex, Do not omit the .bib extension from the filename.
\addbibresource{References/ref_2.bib} %Location of references.bib only for biblatex, Do not omit the .bib extension from the filename.
\renewcommand*{\mkbibacro}[1]{\MakeUppercase{#1}} % to make DOI uppercase
\fi

% ******************************************************************************
% ************************* User Defined Commands ******************************
% ******************************************************************************

% ********************** TOC depth and numbering depth *************************

% \setcounter{secnumdepth}{2}
% \setcounter{tocdepth}{2}

% ******************************* Nomenclature *********************************

\usepackage[intoc]{nomencl}
    \makenomenclature
    \renewcommand{\nomname}{Symbols}

\usepackage[xindy]{imakeidx}
    \makeindex[intoc]
    
\usepackage[xindy,nonumberlist,nopostdot,toc,sort=use]{glossaries}
    \setglossarystyle{long}
    \renewenvironment{theglossary}%
    {\begin{longtable}[l]{lp{\glsdescwidth}}}%
    {\end{longtable}}
    \setlength{\glsdescwidth}{0.8\linewidth}
    \renewcommand*{\glsnamefont}[1]{\textbf{#1}}
    \setacronymstyle{long-short}
    \makeglossaries
    % \glsdisablehyper
    
% ********************************* Appendix ***********************************

% The default value of both \appendixtocname and \appendixpagename is `Appendices'. These names can all be changed via:

%\renewcommand{\appendixtocname}{List of appendices}
%\renewcommand{\appendixname}{Appndx}

% ********************************* Layout ***********************************
% \usepackage{layout}
% \makeatletter
% \renewcommand*{\lay@value}[2]{%
% \strip@pt\dimexpr0.351459\dimexpr\csname#2\endcsname\relax\relax mm%
% }
% \makeatother

% ********************************* Lipsum ***********************************
\usepackage{lipsum}
\usepackage{zhlipsum}
% \usepackage[math]{blindtext}

% *************************** Graphics and figures *****************************
% Specify one or several paths in which to search for figures.
% Don't miss the last "/".
\graphicspath{{Figures/}{Figures/Chapter1/}{Figures/Chapter2/}{Figures/Chapter3/}{Figures/Chapter4/}{Figures/Chapter5/}{Figures/Chapter6/}}
\usepackage{pgf,tikz}

% ******************************************************************************
% ************************ Thesis Information & Meta-data **********************
% ******************************************************************************

% Use \texorpdfstring for PDF metadata-friendly title. Usage:
%\texorpdfstring{LaTeX_Version}{PDF Version (non-latex)} eg.,
%\texorpdfstring{$\sigma$}{sigma}

%% The title of the thesis
\title{A Discussion on Higher Education Development in Hong Kong}
\titlezh{論香港高等教育的發展}
% \shorttitle{Something}

%% The full name of the author
\author{Chan Tai Man}
\authorzh{陳大文}

%% University
\university{City University of Hong Kong}
\universityzh{香港城市大學}
\universityabbr{CityU}
% Joint PhD Programme
% \partneruniversity{University of Science and Technology of China}

%% Department
\dept{Department of Public Policy}
\deptzh{公共政策學系}

%% Full title of the Degree
\degreetitle{Doctor of Philosophy} % Master/Doctor of Philosophy as appropriate
\degreetitlezh{哲學博士學位} % 哲學碩士/博士學位 as appropriate
\degreetitleabbr{PhD} % MPhil or PhD as appropriate

%% Submission date
% Default is set as {\monthname[\the\month]\space\the\year}
\degreedate{January 2021}
\degreedatezh{二零二一年一月}

%% Meta information will appear in the PDF meta-information
\subject{\hologo{LaTeX}} \keywords{{\hologo{LaTeX}} {PhD Thesis} {Engineering}}

% ******************************** Front Matter ********************************
\begin{document}
\VerbatimFootnotes

\frontmatter

\maketitle
% ******************************* Thesis Dedidcation ********************************

\begin{dedication}

To

\end{dedication}


% ************************** Thesis Abstract *****************************

\begin{abstract}
AAA
\end{abstract}

% \section*{CITY UNIVERSITY OF HONG KONG\\ Qualifying Panel and Examination Panel}
% \fancyhf{}
\thispagestyle{plain}

\begin{center}
    {\Large CITY UNIVERSITY OF HONG KONG\\ Qualifying Panel and Examination Panel}
\end{center}

{ % begin box to localize effect of arraystretch change
% \renewcommand{\arraystretch}{1.2}
\begin{table}[H]
\setlength{\tabcolsep}{0pt}
\begin{tabular}{ll}
Surname:                          & Pan                                                                                                                                \\
First name:                       & Fei                                                                                                                                \\
Degree:                           & PhD                                                                                                                                \\
Department:                       & Department of Biomedical Engineering                                                                                       \\
                                  &                                                                                                                                    \\
\multicolumn{2}{l}{The Qualifying Panel of the above student is composed of:}                                                                                           \\
\textit{Supervisor(s)}               &                                                                                                                                    \\
Professor    & \begin{tabular}[c]{@{}l@{}}Department of Biomedical Engineering,\\ City University of Hong Kong\end{tabular}                       \\
                                  &                                                                                                                                    \\
\textit{Qualifying Panel Member(s)} &                                                                                                                                    \\
Professor                  & \begin{tabular}[c]{@{}l@{}}Department of Biomedical Engineering,\\ City University of Hong Kong\end{tabular}                       \\
Professor                   & \begin{tabular}[c]{@{}l@{}}Department of Biomedical Engineering,\\ City University of Hong Kong\end{tabular}                       \\
                                  &                                                                                                                                    \\
\multicolumn{2}{l}{The thesis has been examined and approved by the following examiners:}                                                                              \\
Professor                        & \begin{tabular}[c]{@{}l@{}}Department of Biomedical Engineering,\\ City University of Hong Kong\end{tabular}                       \\
Professor                    & \begin{tabular}[c]{@{}l@{}}Department of Biomedical Engineering,\\ City University of Hong Kong\end{tabular}                       \\
Professor                     & \begin{tabular}[c]{@{}l@{}}Department of Biomedical Engineering,\\ City University of Hong Kong\end{tabular}                       \\
Professor               & \begin{tabular}[c]{@{}l@{}}Department of Mechanical and Automation Engineering,\\ The Chinese University of Hong Kong\end{tabular}
\end{tabular}
\end{table}} 
% \includepdf{Front/panel_sheet.pdf} % the SGS will provide you a new panel sheet
% ************************** Thesis Acknowledgements **************************

\begin{acknowledgements}

It has been a long and thrilling journey, with unimaginable challenges, both academic and social, since I came to CityU in the peaceful summer of 2016.

\end{acknowledgements}


% *********************** Adding TOC and List of Figures ***********************

\tableofcontents
\listoffigures
\listoftables
\cleardoublepage % or \clearpage under the oneside option
\phantomsection
\addcontentsline{toc}{chapter}{List of Algorithms}
\listofalgorithms
\listoflistings

%!TEX root = ../thesis.tex

\nomenclature[01]{$\vv{OP}$}{Displacement of the micropipette tip $P$ to the origin $O$}
\nomenclature[02]{$(a,b,c)$}{Coordinates of $\vv{OP}$ in the world coordinate frame $(O, x, y, z)$}
\printnomenclature[3cm]

%!TEX root = ../thesis.tex

\newacronym{pbs}{PBS}{phosphate buffered saline} 
\newacronym{bp}{bp}{base pair}
\newacronym{ncbi}{NCBI}{National Center for Biotechnology Information}
\newacronym{rna}{RNA}{ribonucleic acid}
\newacronym{mrna}{mRNA}{messenger RNA}
\newacronym{dna}{DNA}{deoxyribonucleic acid}
\newacronym{cdna}{cDNA}{complementary DNA}
\newacronym{crispr}{CRISPR}{clustered regularly interspaced short palindromic repeats}
\newacronym{tgv}{TGV}{train à grande vitesse}
\printglossary[title={Acronyms}]

% ******************************** Main Matter *********************************
\mainmatter
\chapter{Introduction}
\label{ch1:Introduction}

English. \foreignlanguage{russian}{Русский}\footnote{Need to use \verb|FreeSerif| or \verb|Times New Roman| font. \verb|newtxtext| and \verb|newpxtext| don't contain Cyrillic.}. \foreignlanguage{spanish}{Español}.

\Gls{pbs}, \gls{ncbi}, \gls{rna}, \gls{dna}, \gls{mrna}, \gls{cdna}, \gls{crispr}, and \gls{tgv}.

\Gls{pbs}, \gls{ncbi}, \gls{rna}, \gls{dna}, \gls{mrna}, \gls{cdna}, \gls{crispr}, and \gls{tgv}.

\section{Background}
\label{ch1:sec:Background}

Inline math $\Vec{y}=f(\vv{x})$

\begin{equation}
    \vv{E}_{\vv{u}_{\vv{u}}}
\end{equation}

\begin{equation}\label{eq:joint_probability_distribution_function}
  F\left( x,y \right) =\int_{-\infty}^x{\int_{-\infty}^y{f\left( u,v \right) \text{d}u\text{d}v}}
\end{equation}

\begin{align}\label{eq:myalign}
  F_X\left( x \right) =P\left( X\leqslant x \right) & =P\left( X\leqslant x,Y<+\infty \right) \\
                                                    & =\int_{-\infty}^x{\int_{-\infty}^{+\infty}{f\left( u,v \right) \text{d}u\text{d}v}} \\
                                                    & =\int_{-\infty}^x{\left[ \int_{-\infty}^{+\infty}{f\left( u,v \right) \text{d}v} \right] \text{d}u}\\
                                                    & =\int_{-\infty}^x{f_X\left( x \right) \text{d}u}
\end{align}

\begin{equation}\label{eq:myeq}
  \begin{split}
  F_X\left( x \right) =P\left( X\leqslant x \right) & =P\left( X\leqslant x,Y<+\infty \right) \\
                                                    & =\int_{-\infty}^x{\int_{-\infty}^{+\infty}{f\left( u,v \right) \text{d}u\text{d}v}} \\
                                                    & =\int_{-\infty}^x{\left[ \int_{-\infty}^{+\infty}{f\left( u,v \right) \text{d}v} \right] \text{d}u}\\
                                                    & =\int_{-\infty}^x{f_X\left( x \right) \text{d}u}
  \end{split}
\end{equation}

\begin{subequations}\label{euler_rotation_matrix}
\begin{align}
	R_{11}=&\cos \alpha \cos \gamma -\cos \beta \sin \alpha \sin \gamma\\
	R_{12}=&\sin \alpha \cos \gamma +\cos \beta \cos \alpha \sin \gamma\\
	R_{13}=&\sin \beta \sin \gamma\\
	R_{21}=&-\cos \alpha \sin \gamma -\cos \beta \sin \alpha \cos \gamma\\
	R_{22}=&-\sin \alpha \sin \gamma +\cos \beta \cos \alpha \cos \gamma\\
	R_{23}=&\sin \beta \cos \gamma\\
	R_{31}=&\sin \beta \sin \alpha\\
	R_{32}=&-\sin \beta \cos \alpha\\
	R_{33}=&\cos \beta
\end{align}
\end{subequations}

\begin{subequations}
\begin{multline}\label{formula_a_prime_prime}
a''=\frac{\cos \alpha \cos \gamma -\cos \beta \sin \alpha \sin \gamma}{\cos \theta}a\\
+\frac{\sin \alpha \cos \gamma +\cos \beta \cos \alpha \sin \gamma}{\cos \theta}b\\
+\frac{\sin \beta \sin \gamma}{\cos \theta}c
\end{multline}
\begin{multline}\label{formula_b_prime_prime}
b''=a\left( -\cos \alpha \sin \gamma -\cos \beta \sin \alpha \cos \gamma \right)\\
+b\left( -\sin \alpha \sin \gamma +\cos \beta \cos \alpha \cos \gamma \right)\\
+c\left( \sin \beta \cos \gamma \right)
\end{multline}
\begin{multline}\label{formula_c_prime_prime}
c''=\left( \cos \beta -\sin \theta \sin \beta \sin \gamma \right) c\\
-\left(-\sin \beta \cos \alpha -\sin \theta \left( \sin \alpha \cos \gamma +\cos \beta \cos \alpha \sin \gamma \right) \right) b\\
+\left( \sin \beta \sin \alpha -\sin \theta \left( \cos \alpha \cos \gamma -\cos \beta \sin \alpha \sin \gamma \right) \right) a
\end{multline}
\end{subequations}

\section{Statement of Problems}
\label{ch1:sec:StatementofProblems}

\section{Research Objectives}
\label{ch1:sec:ResearchObjectives}

In brief, the research objectives of this study are enumerated as follows:\index{research}
\begin{enumerate}
    \item AAA;
    
    \item BBB;
    
    \item CCC.
\end{enumerate}

\section{Methodologies and Significance}
\label{ch1:sec:MethodologiesandSignificance}

The major methodologies and their significance are discussed below.
\begin{enumerate}
    \item AAA.

    \item BBB.

    \item CCC.
\end{enumerate}

自定義編號和siunitx宏包

\begin{enumerate}[label=\fbox{\arabic*}]
    \item 你
    \item 我
    \item 他
    \item \SI{1.0}{\milli\meter\per\kilo\gram}
\end{enumerate}

\section{Summary}
\label{ch1:sec:Summary}

\chapter{Literature Review}
\label{ch2:LiteratureReview}

\section{Introduction}
\label{ch2:sec:Introduction}

Please refer to the following \hologo{LaTeX} documents \cite{Gai_2005_LaTeXKeJiWenDangPaiBan,Gai_2006_LaTeXMathematicsCompanion,Ghaffari_2014_PreparingFiguresMatlab,Huang_2013_LeiTaiHePaiBanXiTongJianJieLaTeXNotes,Liu_2013_LaTeXRuMen,Oetiker_2019_YiFenBuTaiJianDuanDeLaTeX2eJieShao,Reckdahl_2017_LaTeX2eChaTuZhiNan}.

And other examples \cite{Beck_2018_FirstCourseComplex,Ben-Arieh_2003_TransformationsGeneralizedATSP,Eberly_2019_LeastSquaresFitting,Strang_2016_IntroductionLinearAlgebra}.

\begin{figure}[!htb]
  \centering
  \includegraphics[width=0.9\linewidth]{Figures/Chapter2/ctanlion.pdf}
  \caption{caption}\label{fig:fig1}
\end{figure}

\begin{figure}[!htb]
    \centering
    \subfloat[]{\includegraphics[width=0.4\linewidth]{example-image-plain}}

    \subfloat[]{\includegraphics[width=0.4\linewidth]{example-image-empty}}
    \caption{caption.}
    \label{fig:fig2}
\end{figure}

\begin{figure}[!htb]
  \centering
  \subfloat[]{\includegraphics[width=0.9\linewidth]{example-image-a}}

  \subfloat[]{\includegraphics[width=0.45\linewidth]{example-image-b}}
  \hfill
  \subfloat[]{\includegraphics[width=0.45\linewidth]{example-image-c}}
  \caption{caption}\label{fig:fig3}
\end{figure}

\begin{mdframed}
AAABBBCCC
\end{mdframed}

\section{Summary}
\label{ch2:sec:Summary}
\chapter{Your Main Work}

\section{Introduction}
\label{ch3:sec:Introduction}
\cref{ch2:LiteratureReview} reviews the literature on the following subjects ...

\begin{figure}[ht]
\centering
\begin{minipage}[t]{0.48\linewidth}
  \centering
  \includegraphics[width=\linewidth]{Figures/Chapter3/ctanlion.pdf}
  \captionof{figure}{caption.}
  \label{fig:fig4}
\end{minipage}
\hfill
\begin{minipage}[t]{0.48\linewidth}
  \centering
  \includegraphics[width=\linewidth]{Figures/Chapter3/ctanlion.pdf}
  \captionof{figure}{caption.}
  \label{fig:fig5}
\end{minipage}
\end{figure}

\section{Conclusions}
\label{ch3:sec:Conclusions}

\chapter{Experiments}
\label{ch6:Experiments}

\section{Introduction}
\label{ch6:sec:Introduction}

\section{Experimental Results}
\label{ch6:sec:Experimental Results}

\subsection{Some Results}
\label{ch6:subsec:Some Results}

\subsection{Discussion}
\label{ch6:subsec:Discussion}

\section{Conclusions}
\label{ch6:sec:Conclusions}
\chapter{Conclusions}
\label{ch7:Conclusions}

\section{Summary}
\label{ch7:sec:Summary}

\section{Future Work}
\label{ch7:sec:FutureWork}


% ********************************** Back Matter *******************************
% Backmatter should be commented out, if you are using appendices after References
% \backmatter

% ********************************** Bibliography ******************************
\begin{spacing}{0.9}
\printbibliography[heading=bibintoc, title={References}]
\end{spacing}

\begin{spacing}{0.9}
\printindex
\end{spacing}
% ********************************** Appendices ********************************

\begin{appendices} % Using appendices environment for more functionality
%!TEX root = ../thesis.tex

\chapter{Research Output}

\begin{enumerate}
    \item AAA
\end{enumerate} 
%!TEX root = ../thesis.tex

\chapter{Curriculum Vitae}

\begin{table}[H]
% \setlength{\tabcolsep}{0pt}
\begin{tabular}{lll}
\multicolumn{3}{l}{Personal Data}                                               \\
                &                                &                              \\
Name:           & PAN Fei                        &                              \\
Date of birth:  & January 1, 1000                &                              \\
\end{tabular}
\end{table} 
\end{appendices}

\end{document} 